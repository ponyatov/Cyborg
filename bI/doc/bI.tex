\secrel{Язык \bi}\secdown

\secly{Принципы языка}

\begin{itemize}

\item динамический

\item объектный

\item параллельный
\\ при разработке языка большое внимание уделяется обеспечению параллелизма в 
самом широком смысле: микропотоки, использование потоков runtime-системы и/или
ОС, управление выполнением программ на гетерогенных вычислительных кластерах, 
облачных севисах и p2p распределенных сетях, средства платформенно-независимой
сериализаци, поддержка персистентности и резервирования, синтаксическая 
поддержка языком параллельного программирования. 

\item run-time спецификация типов и оптимизация

\item компилируемый через трансляцию
\\ результирующая система может быть оттранслирована\note{ограниченно}\ в 
исходный код на \cpp\ для обеспечения переносимости программ для систем, для
которыйх не реализована \bi-машина.

\item литературное программирование\note{literate programming}\ и 
автоматическая генерация документации  

\end{itemize}

\secly{Комментарии}

\secrel{Система типов}\secdown

\secrel{Примитивные}\secdown
\secrel{int: целые}
\secrel{float: дробные}
\secrel{string: строки}
\secrel{ptr: указатели}
\secrel{bit: битовые строки}

Были введены для работы с низкоуровневыми машинными целыми и упакованными
двоичными данными, в т.ч. использования в качестве бинарных масок. 

\secup

\secrel{Составные типы и контейнеры}\secdown
\secrel{rec: записи}
\secrel{pair: пары}
\secrel{list: списки}
\secrel{dict: хэш-массивы}
\secrel{graph: деревья и графы}
\secrel{queue: очередь}
\secrel{stack: FIFO}

\secup

\secrel{Математические}\secdown

\secrel{complex}
\secrel{vector}
\secrel{matrix}

\secup

\secrel{Файловые и сетевые типы}\secdown

\secrel{bin: двоичный файл}
\secrel{txt: plain text текстовый файл}
\secrel{file: генерализованные файлы}

Универсальный файл с набором кодеков, лексическим/синтаксическим анализатором, 
динамической структурой, обработчиками событий,.. 
 
\secrel{socket: платформенные сокеты}

\secrel{codec: потоковый конвертер форматов данных}

\secup

\secrel{Типы обеспечения параллелизма}\secdown

\secrel{thread: нить}
\secrel{mutex}
\secrel{semaphore}
\secrel{node: вычислительный узел (ядро процессора)}
\secrel{group: группа узлов}
\secrel{msg: сообщение}
\secrel{prio: приоритет}
\secrel{pipe: межпроцессное соединение}
\secrel{share: шара (общая структура данных)}
\secrel{task: задача}
\secrel{sched: планировщик}

\secup

\secrel{Пользовательский интерфейс}\secdown

\secrel{console}
\secrel{window}
\secrel{menu}
\secrel{text}
\secrel{widget}

\secup

\secrel{Геометрические и САПР типы}\secdown

\secrel{layer}
\secrel{point}
\secrel{line: линия}
\secrel{arc: дуга}
\secrel{nurbs: кривая Безье}
\secrel{color: цвет}
\secrel{facet: полигон}
\secrel{body: тело}
\secrel{dim: геометрический размер}
\secrel{mat: материал}
\secrel{camera}
\secrel{tool: инструмент}
\secrel{machine: станок}
\secrel{detail}
\secrel{assembly}
\secrel{technology}
\secrel{sheet}
\secrel{view}

\secup

\secup

\secup
