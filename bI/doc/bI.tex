\secrel{Язык \bi}\secdown

\begin{itemize}

\item динамический
\\ в LISP-смысле: программа сама является структурой данных, и может
модифицировать другие программы или сама себя, создавать и удалять программы в
процессе выполнения. Так как очень часто приходиться работать с данными в
текстовых форматах, в ядре языка предусмотрен функционал создания парсеров, 
оптимизаторов, трансляторов и компиляторов любых других языков.

\item объектный

\item параллельный
\\ при разработке языка большое внимание уделяется обеспечению параллелизма в 
самом широком смысле: микропотоки, использование потоков runtime-системы и/или
ОС, управление выполнением программ на гетерогенных вычислительных кластерах, 
облачных севисах и p2p распределенных сетях, средства платформенно-независимой
сериализаци, поддержка персистентности и резервирования, синтаксическая 
поддержка языком параллельного программирования. 

\item run-time спецификация
\\ объектов в процессе выполнения программы не предусматривет предварительное
определение объектов, при попытке использования несуществующего объекта
открывается интерактивная отладочная сессия

\item интерактивная отладка
\\ в стиле SmallTalk позволяет программисту создавать программу в диалоговом
режиме в процессе отладки и прогона на тестовом стенде или наборе юнит-тестов.

\item компилируемый через трансляцию
\\ результирующая система может быть оттранслирована\note{ограниченно}\ в 
исходный код на \cpp\ для обеспечения переносимости программ для систем, для
которых не реализована \bi-машина, недостаточно аппаратных 
ресурсов\note{встраиваемые системы}, или
предъявляются жесткие требования по надежности\note{$\uparrow$}.

\item литературное программирование\note{literate programming}\ и 
автоматическая генерация документации  

\end{itemize}

\secly{Комментарии}

\verb|//|


\secrel{Система типов}\secdown

.

\secrel{Примитивные}\secdown

\secrel{\class{int}: целые}

\verb|a = 123|

\verb|b = int("-567")|


\secrel{\class{float}: дробные}

\verb|a = 12.34|

\verb|b = float("-5.6e-7")|


\secrel{\class{string}: строки}

\verb|a = 'одиночные "кавычки"'|

\verb|b = "двойные 'кавычки'"|

\verb|c = "используйте \\"\tдля квотирования \" и спец.символов\x07\n"|


\secrel{\class{ptr}: указатели}
\secrel{\class{bit}: битовые строки}

Были введены для работы с низкоуровневыми машинными целыми и упакованными
двоичными данными, в т.ч. использования в качестве бинарных масок.

\secrel{\class{error}: исключение} 

\secup

.

\secrel{Составные типы и контейнеры}\secdown
\secrel{\class{rec}: записи}
\secrel{\class{pair}: пары}
\secrel{\class{list}: списки}
\secrel{\class{dict}: хэш-массивы}
\secrel{\class{graph}: деревья и графы}
\secrel{\class{queue}: очередь}
\secrel{\class{stack}: FIFO}

\secup

.

\secrel{Математические}\secdown

\secrel{complex}
\secrel{vector}
\secrel{matrix}

\secup

.

\secrel{Финансовые}\secdown
\secrel{currency}
\secrel{tax}
\secup

.

\secrel{Файловые и сетевые типы}\secdown

\secrel{\class{bin}: двоичный файл}
\secrel{\class{txt}: plain text текстовый файл}
\secrel{\class{log}: отладочный лог}

\class{log} обеспечивает запись отладочной информации в процессе выполнения
программ, поддерживаются текстовые файлы, сетевое логирование и передача данных
в интерактивный отладчик.

\secrel{file: генерализованные файлы}

Универсальный файл с набором кодеков, лексическим/синтаксическим анализатором, 
динамической структурой, обработчиками событий,.. 
 
\secrel{socket: платформенные сокеты}

\secrel{codec: потоковый конвертер форматов данных}

\secrel{protocol: реализация сетевого протокола}

\secup

.

\secrel{Типы обеспечения параллелизма}\secdown

\secrel{thread: нить}
\secrel{mutex}
\secrel{semaphore}
\secrel{node: вычислительный узел (ядро процессора)}
\secrel{group: группа узлов}
\secrel{msg: сообщение}
\secrel{prio: приоритет}
\secrel{pipe: межпроцессное соединение}
\secrel{share: шара (общая структура данных)}
\secrel{task: задача}
\secrel{sched: планировщик}

\secup

.

\secrel{Пользовательский интерфейс}\secdown

.

\secrel{\class{dislay}: дисплей}

.

\secrel{\class{hid}: контроллер ввода}

HID: Human Inteface Device

\begin{itemize}[nosep]
\item клавитура
\item мышь
\item тач
\item джойстик
\item MIDI
\end{itemize}

\secrel{\class{console}: текстовая консоль}

.

\secrel{\class{form}: графическая форма, виджет}
.

\secrel{\class{window}: окно}
.

\secrel{\class{text}: текст}

.

\secrel{\class{canvas}: область графического ввода/вывода}

.

\secrel{\class{selector}: выбор элементов (меню, радиокнопки)}

.

\secrel{\class{tree}: графическое дерево}

.

\secrel{\class{table}: таблица}

.

\secrel{\class{plot}: график}

.

\secrel{\class{font}: шрифт}
.

\secrel{\class{status}: статус-элемент (строка состояния, градусник, лампа)}
.

\secrel{\class{placer}: оконный менеджер или расстановщик элементов GUI}

\secup

.

\secrel{Геометрические и САПР типы}\secdown

\secrel{layer}

.

\secrel{point}
.

\secrel{line: линия}
.

\secrel{arc: дуга}
.

\secrel{nurbs: кривая Безье}
.

\secrel{color: цвет}
.

\secrel{facet: полигон}
.

\secrel{body: тело}
.

\secrel{dim: геометрический размер}
.

\secrel{mat: материал}
.

\secrel{camera}
.

\secrel{tool: инструмент}
.

\secrel{machine: станок}
.

\secrel{detail}
.

\secrel{assembly}
.

\secrel{technology}
.

\secrel{sheet}
.

\secrel{view}

\secup

.

\secup

\secup
